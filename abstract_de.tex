\documentclass[../main.tex]{subfiles}

\begin{otherlanguage}{ngerman}
    \begin{abstract}

        \noindent Die Einführung von \Gls{cloud}- und \gls{devops}-Modellen ist aktuell ein Trend in der Industrie.
        Viele grosse Unternehmen führen eine \glsdisp{hybrid_cloud}{Hybrid-Cloud}-Umgebung ein, als Teil der digitalen Transformation.
        Gleichzeitig muss die betriebliche Kontinuität sichergestellt sein.
        Durch die vermehrte Nutzung von \glsdisp{hybrid_cloud}{Hybrid-Cloud} entstehen neue Herausforderungen, welche neue Lösungskonzpte zur Handhabung verlangen.
        In dieser Bachelorarbeit wird eine Lösung für die Handhabung einer \glsdisp{cross_cloud}{Cross-Cloud}-Umgebung auf Basis von \gls{kubernetes} eingeführt.
        Diese arbeitet mit Kennzeichnugen und Strategien für die Platzierung der Komponenten in die jeweilige \Gls{cloud}-Umgebung während des Bereitstellungsprozesses.
        Im Weiteren wird eine Lösung für die Integration der bestehenden Applikationslandschaft bereitgestellt, welche der gleichen deklarativen Vorgehensweise folgt.
        Zur Eignungsprüfung werden Experimente durchgeführt und die Anwendbarkeit wird duch die Einschätzung einer kleinen Gruppe von Experten aus der Industrie ermittelt.
        Die Resultate demonstrieren eine High-End-Leistungsfähigkeit der Lösung und zeigen auf, dass das Potential für eine solche Anwendung in der Industrie vorhanden ist.
        Für die Umsetzung muss die Lösung jedoch erweitert werden, um den Abnahmenormen eines regulierten Betriebes gerecht zu werden.

    \end{abstract}
\end{otherlanguage}

