\documentclass[../main.tex]{subfiles}

\begin{document}

    An assessment with an n=7 expert group has been conducted.
    First, the experience level and background of the expert group is evaluated, followed by an assessment of the applicability of the concepts based on feedback from the group.

    \subsubsection{Experience level of expert group}

    The expert group consists of \acrshort{it} professionals from the finance, telecommunication, software and \gls{cloud} services industry and work as software, platform and \gls{devops} engineer, \gls{cloud} architect and middleware specialist (Fig.~\ref{fig:res_group_bg}a-b).

    For \gls{cloud} environments, \gls{public_cloud} is the predominant area of expertise, while over 50\% have no explicit experience working on or operating a \gls{hybrid_cloud} (Fig.~\ref{fig:res_group_skill}a).
    Skill levels are high on non-\gls{cloud_native}, legacy applications and for containers, whereas skills on \gls{kubernetes} are still developed (Fig.~\ref{fig:res_group_skill}b).
    The years of experience working in \acrshort{it} ranges from 8 to 25 years (Fig.~\ref{fig:res_group_skill}c).

    The most applied \gls{cloud} deployment models are \glsdisp{public_cloud}{public} and \gls{private_cloud}, with \glsdisp{hybrid_cloud}{hybrid}- and \gls{multi_cloud} only making up a small percentage.
    In terms of \gls{cloud} providers used, \gls{microsoft_cloud}, \acrlong{aws} and \acrlong{gcp} are taking the lead, followed by Swisscom, IBM and Hetzner (Fig.~\ref{fig:res_group_bg}c-d).

    Considering tooling for management and deployment, legacy applications are mainly handled with company-internal tools in combination with \gls{puppet}, \gls{ansible} and \gls{terraform} for configuration and infrastructure management.
    For \gls{cloud_native} applications, more open, standardised software and less custom tooling is used with \gls{kubernetes}-based solutions making up more than 50\% of the share, predominantly in combination with \gls{terraform} for infrastructure management (Fig.~\ref{fig:res_group_bg}e-f).

    \subfile{results-fig-group-bg}

    \subfile{results-fig-group-skill}

    % print group figured before the applicability part
    \clearpage

    \subsubsection{Applicability of legacy integration}

    The applicability of integrating legacy applications into the deployment workflow using \gls{kubernetes} manifests is evaluated based on usage consideration and \acrlongpl{kpi}.
    Usage consideration mainly received a neutral or strong positive rating.
    For \acrlongpl{kpi}, both software development and incident management are mostly estimated to become more efficient, however, a minority thinks that it will have a strong negative impact on incident management cost and efficiency (Fig.~\ref{fig:res_assess_legacy}a-d).

    \subfile{results-fig-assess-legacy}

    There is a positive correlation between a lower experience level with \gls{kubernetes} and \gls{cloud} environments and a higher overall rating score.
    This mainly reflects the part of the group that is very experienced with legacy applications in a large enterprise.

    Further feedback that supplements and manifests the ratings in Fig.~\ref{fig:res_assess_legacy} is aggregated and presented in consolidated form.
    The main points for legacy integration from the rating justification and additional input are:
    \begin{itemize}
        \setlength\itemsep{0em}
        \item \textquote{Work with legacy applications is not in scope}
        \item \textquote{Redesign of legacy applications should be favoured}
        \item \textquote{Amount of tooling for deployment should be kept to a minimum}
        \item \textquote{Avoid custom tooling as maintenance causes additional overhead}
        \item \textquote{Consistent process will simplify support}
        \item \textquote{Consistent process will help migrating off the legacy stack}
        \item \textquote{Integration would be needed as migrating all processes to the \gls{cloud} is not possible}
        \item \textquote{Integration would help to ease and speed up \acrshort{ci}/\acrshort{cd}}
        \item \textquote{Integration would decrease lead time and cost}
        \item \textquote{Hard to convince senior management to adopt a new process for a legacy stack}
    \end{itemize}

    \subsubsection{Applicability of hybrid cloud policies}

    Separately from legacy integration, applicability of \gls{hybrid_cloud} deployments using tags and policies is also evaluated on the same basis of usage consideration and \acrlongpl{kpi}.
    There is a strong positive usage consideration and performance in terms of cost and efficiency has been mostly estimated to improve, strongly improve or at least stay the same (Fig.~\ref{fig:res_asses_policy}a-d).

    \subfile{results-fig-assess-policy}

    Further feedback that supplements and manifests the ratings in Fig.~\ref{fig:res_asses_policy} is aggregated and presented in consolidated form.
    The main points for \gls{hybrid_cloud} deployments from the rating justification and additional input are:
    \begin{itemize}
        \setlength\itemsep{0em}
        \item \textquote{Simple way to manage multiple \gls{cloud} environments}
        \item \textquote{Current \gls{cloud} policy can be easily accessed using \gls{git}}
        \item \textquote{Helps to avoid vendor lock-in}
        \item \textquote{Brings consistency across environments}
        \item \textquote{Reduces overhead of the time to deploy}
        \item \textquote{Easy correlation of deployments and outages, if related}
        \item \textquote{Helps to meet high availability / business continuity planning requirements}
        \item \textquote{Might not match the real-world complexity of deployments}
    \end{itemize}

    On more general terms, looking at the overall solution, additional input has been provided as what should be considered for future work to improve applicability:
    \begin{itemize}
        \setlength\itemsep{0em}
        \item \textquote{Integrate with an approval process for release to production systems}
        \item \textquote{Use and extend existing \gls{gitops} workflow tools}
        \item \textquote{Use \gls{kustomize} for \gls{kubernetes} configuration management}
    \end{itemize}

\end{document}

