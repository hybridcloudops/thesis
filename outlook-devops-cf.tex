\documentclass[../main.tex]{subfiles}

\begin{document}

    Every change is an opportunity to learn for software developers, system administrators and product managers, if sufficient information is provided about the behaviour of the system and their users.
    Unlike most \acrlong{sdlc} processes, \gls{devops} is visualized as a loop, transitioning from monitoring to planning as part of \gls{cont_feed}.
    It is important to acknowledge and promote continuous learning and experimentation, a core principle of \gls{devops}, as part of the lifecycle.
    With the knowledge of how the systems behave and how the clients use them, more educated decisions can be made that drive the development of products.
    To gain that knowledge, the relevant data has to be gathered and made broadly accessible.

    This includes data from all \gls{devops} stages and not only metrics from the execution environments.
    Quality metrics from static code analysis and security vulnerability scanning should come from the \acrlong{ci} process.
    The overall lead time, deployment lead time, change frequency and change failure rates can be tracked as part of the development process.
    Runtime monitoring and incident management should provide \acrlong{mtta} an incident, \glsdisp{mttrc}{mean time to recover} from it, \acrlong{mttrs} a problem, \acrlong{mttf} and \glsdisp{mtbf}{between failures} and \acrlong{mdt}.
    To estimate the profitability, revenue, return of investment of changes and \acrlong{tco}, combining \gls{cap_ex} and \gls{op_ex}, should be calculated.
    More challenging to track but equally important is employee happiness and satisfaction.
    Combining all metrics into a knowledge base enables data-driven \gls{cont_feed}.

    By including all the feedback, planning can be effective.
    For development to be equally effective, the stability of production systems has to be ensured.
    Developers having to track down production issues kills productivity and motivation.
    The current operational status of the systems has to be visible at any point in time, notifying proactively about potential problems.
    The standard with \gls{kubernetes} is to use liveliness and readiness probes.
    This requires applications to expose health endpoints that provide a detailed picture of the current state of runtime dependencies.
    With real-time monitoring of those endpoints in combination with log processing, making accurate predictions and raising alerts on irregularities is possible.
    Altogether, this forms the \gls{hybrid_cloud_ops} dashboard, which provides access to all the information.

\end{document}

