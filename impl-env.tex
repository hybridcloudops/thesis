\documentclass[../main.tex]{subfiles}

\begin{document}

    The reference \gls{hybrid_cloud} environment is composed of three parts.
    The \gls{public_cloud} part, running in a selected region on commodity hardware of a \gls{public_cloud} provider, the \gls{private_cloud} part and the legacy stack, both of which are running on my personal device (Fig.~\ref{fig:env_simple}).

    \subfile{impl-fig-env-setup}

    \subsubsection{Azure Kubernetes Service}

    I am focusing on \gls{microsoft_cloud} as the \gls{public_cloud} provider given its relative popularity in Switzerland and many well-known Swiss companies using it, including UBS, Raiffeisen, SwissRe and Swisscom.\cite{microsoft_exand_swiss,microsoft_ubs,microsoft_azure_case}

    \gls{terraform} offers an \glsdisp{microsoft_cloud}{Azure} provider, which is used to provide a \gls{kubernetes} cluster via \acrlong{aks}.
    It allows for automated provisioning and destruction of the complete environment.
    The cluster is managed remotely through \gls{kubectl}.

    \subsubsection{Minikube on KVM}

    For my \gls{private_cloud} execution environment, I am using a local \gls{minikube} cluster.
    \Gls{minikube} is designed to run \gls{kubernetes} locally.
    It runs a single-node cluster on top of a virtual machine provisioned through a virtualization driver, in my case \acrfull{kvm}.
    Most \gls{kubernetes} features are available with \gls{minikube} and it can be used with \gls{docker} as the container runtime.
    Cluster interaction is supported using \gls{kubectl}.
    Therefore, the control mechanism is the same as for the \gls{public_cloud} execution environment.\cite{k8s_dist_minikube}

    \subsubsection{Physical linux host}

    My legacy environment is a single, physical linux host.
    Legacy artifacts are not shipped as containers and therefore require the runtime to be available on the machine.
    The artifacts are made available via a web server that serves as an artifact repository.
    To make it possible to deploy legacy components using my legacy manifests, I created a deployment daemon named \gls{legacyctld}, and a command line interface, \gls{legacyctl}.

    \textbf{\gls{legacyctld}} is a server component that manages legacy resources.
    It takes care of applying and deleting \gls{kubernetes} manifests the way \gls{kubectl} does it.
    The server keeps track of all processes that have been started on that instance.
    It offers a \acrshort{rest} \acrshort{api} for process management (Table.~\ref{tab:legacyctld_api}).

    \subfile{impl-tab-legacyctld-api}

    \textbf{\gls{legacyctl}} is a command line interface and \acrshort{http} client of \gls{legacyctld} that handles deployments via \gls{kubernetes} manifests using the legacy extensions.
    It is based on the \acrshort{api} definition of \gls{kubectl}.

\end{document}

