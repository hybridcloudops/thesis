\documentclass[../main.tex]{subfiles}

\begin{document}

    With \gls{kubernetes} being a fundamental part of the \gls{hybrid_cloud} setup, I want to describe all applications and resources with \gls{kubernetes} manifests.
    For big enterprises undergoing digital transformation, having a pure \gls{hybrid_cloud} environment is rarely the case to begin with.
    More realistically, a legacy \acrshort{it} stack is handling a significant fraction of the workload and still requires deployments for updates and patching.
    Therefore, a solution is required to handle deployments to the \gls{hybrid_cloud} and the legacy stack in a standard fashion.

    \subsubsection{GitOps workflow}

    \subfile{concepts-pipeline}

    \subsubsection{Support for legacy deployments}

    Legacy applications are a challenge as containerizing and running them as ephemeral units is often not feasible.
    A re-architecture into \gls{cloud_native} microservices would be a sensible approach from a technical perspective but this might not always be realistic due to economical factors.
    Although this is a prerequisite to run those applications on \gls{kubernetes}, I still want to follow a similar approach for deployments to on-premise servers where the legacy stack is running on.
    Given that \gls{kubernetes} manifests are just information declared in files using a standard format, their use is not necessarily limited to \gls{kubernetes}.
    Therefore, the existing manifests are extended and used as a basis for the legacy deployments.

    To capture all the fields required to perform a deployment, annotations are used (Fig.~\ref{fig:annotations_metamodel}).
    Annotations provide arbitrary metadata about the artifact that is not used for identification, but to provide additional information.\cite{k8_api_annotations}

    \subfile{concepts-fig-meta-annotations}

    Analogous to \gls{kubernetes} that offers a control mechanism through \gls{kubectl}, I am using a command line interface for legacy deployments that talks to an agent running on the respective target host as a daemon process.
    The manifests are then used to run the applications via the agents (Fig.~\ref{fig:depl_legacy}).
    The target host could be a virtual machine or a physical server.
    The agent acts as a gateway for deploying, starting and stopping applications and runs a service registry to keep track of all managed processes.
    Target hosts are expected to be fully operational already.
    Therefore, it does not require any infrastructure provisioning facilities.
    The only purpose is to streamline and unify the deployment process by integrating the existing application landscape.

    \subfile{concepts-fig-deploy-legacy}

    Migrating existing legacy applications to follow a new deployment model will require some initial effort, which is an overhead.
    In contrast, the advantage of this approach is a more standardised application landscape that allows for a less error-prone process due to consistent conventions.
    The learning curve for engineers that are not familiar with the applications is less steep and maintenance will become much more efficient and effective.
    In addition, the migration process can be viewed as a first phase of migrating to the \gls{cloud}.
    Dependencies should be captured already in the manifest files to allow the use of containers and eventual adoption of \gls{kubernetes} at a later stage.

    \subsubsection{Environment support model}

    The use of legacy manifests in combination with the \gls{hybrid_cloud} setup results in three distinct execution environments.
    For effective scheduling during the execution of the \gls{gitops} pipeline, a way to associate and allocate resources is required.

    An application might only support the \gls{public_cloud} environment due to the integration of provider-specific products or infrastructure.
    On the other hand, most stateless services should be able to run in any environment.
    This results in a combination of possible resolutions of support models.
    Resolution sets are formed from all combinations of \gls{public_cloud} $u$, \gls{private_cloud} $r$ and legacy $l$, which results in \gls{hybrid_cloud} $u\land r$, any combination of \gls{cloud} and legacy $u\land l, r\land l$ and the combination of all cases $u\land r\land l$ (Table~\ref{tab:tag_resolution_sets}).

    \subfile{concepts-tab-tag-res}

    An option to associate applications would be to maintain a repository per execution environment.
    However, with this solution it is not possible to monitor a single repository for state changes anymore.
    Also, moving files between repositories does not allow for a clear, auditable history.
    More importantly, a transaction mechanism over all repositories would be required, as it is not otherwise possible to perform a relocation of services between execution environments atomically.
    To illustrate this, imagine a service has to move from the \gls{public_cloud} to the \gls{private_cloud}.
    With \gls{gitops}, the deployment workflow will be triggered once the state change is persisted.
    In this scenario, it would take two separate state changes, triggering two separate deployments.
    Without synchronisation, atomicity cannot be guaranteed and the outcome may not be as expected, as any deployment may fail in the process.

    A support model based on tags provides a flexible way to manage resources.
    By using tags, all manifests can live in a single repository and applications can be organized in groups representing responsibilities or business capabilities, instead of being segregated based on execution environments.
    Also, all state is collocated, resulting in a single source of truth.
    An application can express its need to run on the legacy stack using a specific tag.
    Likewise, at tag is used for \glsdisp{public_cloud}{public} and \gls{private_cloud} support, respectively.
    If an application uses a \gls{public_cloud} service, it needs to be tagged for \gls{public_cloud} only.
    If an application is stateless and self-contained, it can have both the \glsdisp{public_cloud}{public} and \gls{private_cloud} tags.

    To express this, labels on \gls{kubernetes} objects are used that identify which \gls{cloud} environments are supported (Fig.~\ref{fig:labels_metamodel}).
    Labels are intended to be used as metadata to describe and identify objects.
    They should be meaningful to the the user and can be used to query objects and group them into sets.\cite{k8_api_labels}

    \subfile{concepts-fig-meta-labels}

    During the deployment workflow, the deployment algorithm then decides how to schedule applications effectively, based on all available \gls{kubernetes} objects and the tags they have assigned (Algorithm~\ref{alg:deploy_tags}).

    ~\\
    \subfile{concepts-alg-deploy-tags}

    \subsubsection{Policy-driven scheduling}

    While tags express the support model of an application, they do not capture the desired location of execution very well.
    Tags, as per my definition, operate on an application-level.
    To keep concerns separated, I want to use a different instrument to drive deployment decisions on an organisation-level.
    Encoding the information about what is supported and what is resolved in a single value limits the flexibility a \gls{hybrid_cloud} environment has to offer.
    By introducing deployment policies, deployment decisions can be deferred to a later stage of the workflow.

    Policies provide an extensible framework that supports the incorporation of any kind of information for decision making, vertical to the technical capabilities.
    A few possible verticals include regulations, economical or strategic decisions and heuristics.
    Regulations may require application placement in \glspl{private_cloud} for data confidentiality reasons to be compliant with laws.
    Companies may have a \gls{cloud} strategy driven on \acrshort{cto} level that dictates a desired percentage of \gls{public_cloud} use, or sets a maximum to limit dependence.
    Policies could also implement a heuristic approach to place applications based on information about the execution environments, to make decisions geared towards pricing, performance, latency or network load, using \acrshort{ai} in the process.
    To be able to make relevant decisions, models would have to be developed that operate on a curated knowledge base for \gls{cloud} service pricing and time series data of relevant operations metrics, gathered over a period of time.

    Within the scope of this work, I only evaluate the aspect of application placement based on static labels for \gls{cloud} environments, explicitly managed by the user.
    Tags declare the options, policies decide how to make use of them.
    The deployment pipeline evaluates those factors and executes accordingly (Fig.~\ref{fig:pipeline_policies}).

    \subfile{concepts-fig-deploy-polices}

    Based on the resolution sets computed in table~\ref{tab:tag_resolution_sets}, a set of valid decision options arise that have to be evaluated on each deployment.
    The options are: not deploying, deploying to a single environment if applicable with the option of falling back to another environment, or deploying to multiple environments.
    The complexity grows with the number of possible resolutions (Table~\ref{tab:policy_resolutions}).

    \subfile{concepts-tab-policy-res}

    For the sake of simplicity, I assume that the legacy execution environment is never a preferred destination for deployments.
    In other words, if a resource supports \gls{cloud} and legacy, \gls{cloud} will always take precedence.
    That way, legacy components are migrated to a \gls{cloud} execution environment as soon as it is supported, without the need of additional policies.
    Therefore, any resolution that takes into account the legacy stack can be excluded from the list, namely public-legacy, private-legacy and public-private-legacy combinations.

    Based on those types, policies can be developed and associated with applications.
    \gls{kubernetes} uses the concept of namespaces.
    Although namespaces provide a way to build virtual clusters of applications, they are meant to be used to provide separation of concerns for multiple users such as tenants or teams.
    For the level of separation I seek, labels are recommended in the \gls{kubernetes} documentation.
    Therefore, another label is introduced for grouping of applications so that policies can be applied to groups instead of individual applications.\cite{k8s_namespaces}

    The group label has to represent a logical set of applications that serve a particular purpose and follow the same deployment strategy and hence can be associated with a single policy.
    Group labels can have an implicit structure expressing to the user the boundaries they apply.
    As an example, a group of stateless services deployed across \glspl{cloud} might encode high availability in its name and by doing this, provide explicit documentation.

    Putting it all together, a deployment policy associates a group of applications with a set of labels that represent desired execution environments (Fig.~\ref{fig:policies_metamodel}).

    \subfile{concepts-fig-meta-policies}

    With this information, the deployment algorithm (Algorithm~\ref{alg:deploy_tags}), which is based only on tags, can be extended to incorporate policies.
    Given a set of deployments $D = \{ d_i | i \in \mathbb{N} , \text{for all manifests} \}$ with $D_u \cup D_r \cup D_l = D$, if $D_x \subseteq D$ is applied to any execution environment, or any combination of such, the inverse set $D_x'$ has to be deleted from the remaining execution environments to prevent inconsistencies between the desired state captured in the metadata repository and the state applied to the execution environments.
    Deployment strategies have to be idempotent and keep the state unchanged if not applicable (Algorithm~\ref{alg:deploy_policies}).

    ~\\
    \subfile{concepts-alg-deploy-policies}

\end{document}

