\documentclass[../main.tex]{subfiles}

\begin{document}

    \glsdisp{ci}{Continuous integration} is key to efficiently build functional, stable software.
    \Gls{trunk_dev} is the development workflow that has been acknowledged to promote \acrshort{ci}/\acrshort{cd} and \gls{devops}.
    \gls{gitops} works well with \gls{trunk_dev}, but to make proper use of it, a \acrshort{ci} strategy has to be developed.
    In the scope of large enterprises, a development workflow has to scale well.
    For \gls{trunk_dev}, this means using short-lived feature branches with pull-requests to the master branch (Fig.~\ref{fig:git_branching_master}).
    On pull-requests, a pre-integration step and a code review is required to assure the quality of a change.
    Taking the largest \gls{git} repository as reference, this would have to scale up to 2500 pull-requests a day.\cite{trunk_based_dev,largest_git_repo}.

    \subfile{outlook-fig-git-branch-master}

    For this process to be automated, a \acrshort{ci} pipeline has to be put in place that tests the changes and feeds back to the pull-requests.
    Testing in this case is rather complex as it targets multiple execution environments.
    Traditionally, testing environments are set up and used to perform integration testing in an automated fashion but also manually by developers and testers.
    The problem with this approach is that it can become a bottleneck, inconsistencies can develop resulting in environment disparity and it comes with an additional maintenance overhead.
    To avoid this, testing environments should be available on-demand and scale based on the current needs (Fig.~\ref{fig:tes_env_scaled}).

    \subfile{outlook-fig-test-env-scale}

    Starting up a few containers is cheap, but provisioning a full \gls{hybrid_cloud} environment for every testing cycle can become expensive and slow down the integration process.
    The environment could be reduced to one single-node cluster, along the lines of \gls{minikube}, but that would not reflect the real complexity of the system, which is especially problematic when working with highly distributed systems.
    To solve this problem, a virtual environment is built and provisioned in a single container, that reflects the real complexity and can be used for \acrshort{ci} as well as local development.
    This can be achieved by running a local \gls{kubernetes} cluster using \gls{docker} containers as nodes with \gls{kind}, a tool initially designed for testing \gls{kubernetes} itself\cite{kind_k8s_in_docker}, in combination with \gls{mininet} to create the underlying virtual network\cite{mininet}, to provide a realistic \gls{hybrid_cloud} setup.
    All of this then runs in a single container and can be bootstrapped, on-demand, with metadata from the environment repository (Fig.~\ref{fig:tes_env_container}).

    \subfile{outlook-fig-test-env-container}

    The \acrshort{ci} pipeline for testing can run on the \gls{public_cloud}, making full use of the pay-as-you-go model.
    That way, \gls{private_cloud} requirements can be reduced to a minimum, as no additional hardware is required to provision test environments.

    Once a change has passed through the \acrshort{ci} pipeline, it can be promoted to the staging environment.
    At this stage, \acrlong{cd} should take the change to production systems.

\end{document}

