\documentclass[../main.tex]{subfiles}

\begin{document}

    For the assessment, the implementation of the concepts developed as part of this thesis are presented to a small group of selected people that work in the \acrshort{it} industry.
    The group of people, which is referred to as the expert group, is being presented with a video sequence that demonstrates the concepts and implementation of hybrid continuous deployments.
    Before and after that, they are asked to provide feedback based on a questionnaire.

    Before the demonstration, the expert group is asked to answer basic questions that assess their experience level and background.
    No demographic information is required.
    To get an idea of the background of each individual, the current job title, the company they work for and the years of experience working in \acrshort{it} are recorded.
    This information allows to draw conclusions on the relevance of the presented solution for the industry.

    The experience level will be evaluated for the following technologies and concepts based on self-evaluation:
    \begin{itemize}
        \setlength\itemsep{0em}
        \item \Gls{cloud} models (public, private, hybrid, multi)
        \item \Gls{cloud} providers
        \item Containers (\gls{docker})
        \item Container orchestration tools (\gls{kubernetes})
        \item Legacy, non-\gls{cloud_native} applications
    \end{itemize}

    In addition, the expert group will be presented with partially structured questions to find out what technologies are used when working with \gls{cloud_native} or legacy applications, if applicable to them.

    After the demonstration, the expert group will be asked to provide feedback on the applicability of the demonstrated concepts.
    The feedback will be in two parts, one for \gls{cloud} tags and policies and the other for legacy application deployments.
    Both parts follow the same structure, starting with a rating question that assesses the overall applicability, followed by unstructured questions to gather additional feedback on why they would or would not use the demonstrated solutions.
    Also, for each concept it is assessed how the expert group rates it in regards to the \acrlongpl{kpi} for software development and incident management.

\end{document}

