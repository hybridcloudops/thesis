\documentclass[../main.tex]{subfiles}

\begin{document}

    \textbf{HybridCloudOps Models for Business Application Development}

    The current trend in the industry is to move applications to the cloud.
    While public cloud offerings are very effective, big corporations tend to prefer running a part of their stack in a private cloud for various reasons, leaving them with a hybrid cloud environment.
    Hybrid cloud operation is well understood and supported with tools, but software development for hybrid cloud-native applications is still a challenging topic.

    It is generally good practice to use continuous delivery in the field of software engineering following an approach like DevOps.
    The aim of this thesis is thus to elaborate and evaluate concepts for different stages of the DevOps toolchain based on a hybrid cloud enterprise environment, and to arrive at a re-usable definition of a HybridCloudOps concept that involves elements from GitOps and other modern DevOps flavours.

    Expected outcome of the thesis:
    \begin{itemize}
        \item Overview about software development approaches around DevOps, in particular in the field of cloud-native application development, and about cloud landscapes (public-, private-, hybrid-, multi-, cross-cloud)
        \item Concept on how to effectively deploy selected stages of the DevOps toolchain within an exemplary hybrid cloud environment
        \item Proof of concept applied on a hybrid cloud model based on 2 to 3 DevOps stages in detail, including a validation of benefits on a technical level
        \item Recommendations for enterprises migrating to a hybrid cloud model derived from findings in the thesis
    \end{itemize}

\end{document}

