\documentclass[../main.tex]{subfiles}

\begin{document}

    This section shows the actual models and implementation of manifest extensions based on the metamodels of the concepts.

    \subsubsection{Annotations for legacy deployments}

    For legacy manifests I defined a set of annotations that are required by \gls{legacyctld} to fetch artifacts and execute deployments (Fig.~\ref{lst:template_meta}).

    \subfile{impl-manifest-lst-annotation}

    \textbf{legacy/v} specifies the version of the artifact being deployed.
    The version accepts any number and can be independent of the image version.
    Version changes are used to indicate that the artifact needs to be redeployed.

    \textbf{legacy/type} specifies the type of the artifact being deployed.
    Currently the only supported type is \verb|zip|.
    It expects the image to be a \verb|zip| assembly that comes with the execution instructions for the process in form of a run script.

    \textbf{legacy/host} specifies the endpoint of the \gls{legacyctld} process that should take care of deploying the application.

    \textbf{imageregistry} specifies the endpoint of the artifact repository.
    By default, images are fetched from \gls{docker_hub}.
    For legacy deployments, the artifacts need to be fetched from another registry.

    As legacy deployments are not used to deploy containers, the container specification cannot reference an image on \gls{docker_hub}.
    Instead, the image given is derived from the defined image registry based on the annotated artifact type (Fig.~\ref{lst:image_spec}).

    \subfile{impl-manifest-lst-artifact}

    \subsubsection{Environment support labels}

    To indicate support for execution environments, I use labels for legacy, \glsdisp{public_cloud}{public} and \gls{private_cloud}.
    Additionally, I use another label to logically group processes (Fig.~\ref{lst:meta_labels}).

    \textbf{cloud-env-*} indicates if the manifest is describing an artifact that is meant to be running on a \glsdisp{public_cloud}{public} or \gls{private_cloud} environment.
    The label is suffixed with the cluster context it should target and a value of \verb|supported| makes it eligible to run in that environment.
    In my case, the labels are expanded to \verb|cloud-env-minikube| for \gls{private_cloud} and \verb|cloud-env-bsc-aks| for \gls{public_cloud}.

    \textbf{cloud-legacy} indicates if the manifest is describing an artifact that is meant to be running in the legacy environment.
    A value of \verb|supported| indicates that this manifest is designed for the legacy deployment process.

    \textbf{cloud-group} indicates which logical group of services an application belongs to.
    All applications within a group must follow the same deployment strategy.

    \subfile{impl-manifest-lst-label}

    \subsubsection{Hybrid cloud policies}

    Policies are created as \acrfullpl{crd}.
    \acrshortpl{crd} are a way to extend the \gls{kubernetes} cluster \acrshort{api} using \acrshort{api} aggregation.
    They are available from version 1.16.0.
    The outcome is a new kind of \gls{kubernetes} object that can be used with \gls{kubectl} and comes with almost all features available on standard \gls{kubernetes} objects.\cite{k8s_crd}

    A \gls{cloud} policy expects a set of environment labels as part of its specification.
    The default metadata section is populated with the group label, creating the group-to-policy association.
    These labels correspond to the \gls{cloud} labels I defined as part of the concepts (Fig.~\ref{lst:cloud_policy}).

    \subfile{impl-manifest-lst-policy}

    Having a set of \gls{cloud} policies deployed to multiple clusters ensures high availability.
    The policies can be loaded from one cluster, acting as the primary instance, while the other cluster serves as secondary, backup instance.
    \Gls{cloud} policies can be retrieved from the cluster using \gls{kubectl} (Fig.~\ref{lst:kube_getcpol}).

    \subfile{impl-manifest-lst-getcpol}

\end{document}

