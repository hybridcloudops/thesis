\documentclass[../main.tex]{subfiles}

\begin{document}

    The conclusion is drawn from the results based on \acrlongpl{kpi} and further remarks are made based on the assessment feedback.

    \subsubsection{Applicability of concepts}

    The assessment with a small expert group has resulted in an overall positive rating of the applicability of the concepts.
    Every participant would consider using \gls{hybrid_cloud} deployments and a majority of participants would do so in combination with the integration for legacy applications, if required.
    This provides an initial proof of concept.

    \Gls{hybrid_cloud} deployments using a support model based on tagging in combination with policies has been viewed as a simple and effective solution to the presented problem.
    Realistically, this concept on its own is not enough to cover the full release process and it is also not designed for that.
    As part of the assessment, it has been highlighted that real-world deployments can be more complex.
    An important note to make is that the deployment of processes also depends on the design of the application.
    The concept of \gls{hybrid_cloud} deployments is an attempt to build a generic solution for \gls{hybrid_cloud} management and does not take application-specific requirements into account.
    However, together with the support model based on tags, it builds a framework that can be extended to capture additional needs.
    The concepts have been used and proven useful as part of the instrumentation of this thesis, to selectively deploy the monitoring stack for taking measurements.

    Integration of legacy applications into the same workflow using \gls{kubernetes} manifests has been seen as a useful concept, with a few exceptions.
    The main topic of controversy has been the additional effort of adopting the solution versus spending time on the re-architecture and the migration to the \gls{cloud}.
    Also, onboarding of new solutions in large enterprises can be a bureaucratic process.
    For small businesses, this is not an applicable concept, as they often do not maintain a large legacy stack.
    For large enterprises, migrating to a modern architecture should certainly be preferred.
    However, this is not always feasible, economically or technically.
    As an example, some financial institutions have low latency requirements which they have built dedicated infrastructure for.
    The cost of maintaining such an infrastructure is very high but might be part of the business model.
    Migrating the applications built for that infrastructure to commodity hardware is not an option.
    Even if a \gls{cloud} provider offers an equivalent alternative, a return of investment is highly improbable.

    Ideally, everything would run on a single \gls{hybrid_cloud} environment.
    More realistically, there will always be a legacy stack.
    With Microsoft and HashiCorp offering products for management of non-cloud, on-premise applications and infrastructure in combination with their \gls{cloud} stack, there is likely a market for legacy integration.

    \subsubsection{Evaluation of performance indicators}

    The deployment workflow has demonstrated a good performance, which is also related to the use of \gls{gitops} as a basis for the workflow.
    With a deployment time of 55 seconds, as seen in the experimentation with the direct method, a daily deployment frequency of 1570 could be reached.
    This is already above the yearly deployment frequency of high performers in the industry and also allows to reach a deployment lead time of less than a day (Table~\ref{tab:state_devops}).
    The numbers compete with values from elite performers like Amazon, Google and Netflix, that run thousands of deployments a day aggregated over all their services.\cite{state_of_devops_19}

    \subfile{outlook-tab-devops-state}

    Although top performance is a prerequisite for adoption, it is not a unique selling proposition as many alternatives already exist.
    More relevance can be attributed to incident management in terms of failure recovery performance.

    Looking at past events, failure to operate has caused enterprises significant economic loss.
    A 12 hour outage costs Apple 25 million in 2015, a five hour outage costs Delta Airlines 150 million USD in 2016 and a 14 hour outage costs facebook 90 million USD in 2019\cite{atlassian_cost_downtime}.
    Running software in the \gls{cloud} does not prevent such events from happening.
    In 2015, an outage of \acrshort{aws} lasted for five hours\cite{cna_patterns}.
    For \glsdisp{microsoft_cloud}{Azure}, outages have been reported in May 2019 for two hours\cite{azure_out_may_19}, January 2019 for 16 hours\cite{azure_out_jan_19}, November 2018 for 17 hours\cite{azure_out_nov_18} and even for more than a day in September 2018\cite{azure_out_sept_18}.

    While building failure tolerant and stable applications to minimize incidents should be the primary goal, outages cannot be fully avoided.
    In such cases, falling back to an alternative \gls{cloud} using \gls{hybrid_cloud} deployments with \gls{cloud} policies could save a business.
    Experimentation with the phased method demonstrated zero-downtime deployments within 162 seconds.
    For services matching the complexity of the experiment and that can be restored by spinning up a new instance, this would allow a standard business to restore a service with an estimated outage cost of USD 13 500.
    For a large enterprise in the finance industry it would add up to USD 225 000.
    Compared to industry values, the results compete with the high performance group (Table~\ref{tab:outage_cost}).

    \subfile{outlook-tab-outage-cost}

    In order to be able to calculate realistic values for the actual benefit gained based on the formulas for \acrlongpl{kpi}, the probability that an outage occurs and a bug is fixed and the probability that a change request is implemented have to be factored in.
    In an operational organisation, the number of severe outages should be minimal.
    Therefore, the outage probability should be relatively low in comparison to the change probability.
    This puts the impact of the outage cost into perspective.~\eqref{eq:actual_cost}

    \subfile{outlook-eq-cost}

    \subsubsection{Relevance for the IT industry}

    \Gls{hybrid_cloud} has gained a lot of popularity and will likely become the default operating model in large enterprises.
    As a result, the demand will grow not only for pure \gls{hybrid_cloud} management, but for solutions that focus on the full software estate a company needs to operate.
    However, to be relevant for the industry, further work is required to extend the current concepts to fulfill all acceptance criteria of a deployment and release process in a regulated enterprise environment.

    \subsubsection{Relevance for research}

    Optimal placement of applications based on various factors is an ongoing area of research.
    More and more work is done based on container orchestration platforms such a \gls{kubernetes}.
    This work may be used as a basis for research conducted on placement and migration of \gls{kubernetes}-based solutions.

    \subsubsection{Review of the problem statement}

    The goal of this thesis was to develop a \gls{hybrid_cloud_ops} solution that enables the adoption of a \gls{hybrid_cloud} model, following a \gls{devops} approach.
    As part of the work, the target environment was extended to incorporate the legacy \acrshort{it} stack, making it even more hybrid.
    The presented solution sets the basis for hybrid continuous deployments while working with the given hybrid setup.
    The stages of packaging and releasing are the link between build and operate, between Dev and Ops.
    By providing a unified solution for working with a hybrid environment, the inhibition threshold of adopting a cloud solution is reduced and all parties can work with known processes and principles.

    Due to the background of the author, this work was mainly driven by lessons and experiences from the banking business.
    In an organisation with thousands of employees, the biggest challenges are communication and culture.
    However, technical burdens are consuming a lot of time and hinder the culture to thrive.
    Regulatory work, reporting, audit requirements and security measures contribute to the problem.
    Those barriers have to be cut through.
    It is such an environment that can profit the most from the presented solutions, taking it as a basis for streamlining \acrshort{it} processes, allowing the firm to focus on contributing actual value to the economy and society.

\end{document}

