\documentclass[../main.tex]{subfiles}

\begin{document}

    In this section, recommendations are made for enterprises in regards to \gls{cloud} and \gls{devops} based on insights gained from the conducted work and personal experience.

    \subsubsection{Move to the cloud}

    \Gls{cloud} enables everyone to focus on their core business.
    In the end, \gls{cloud} is just an expression to hide all the complex operations-related topics behind a simple term.
    Analogously, this is what applying \gls{cloud} concepts means for an \acrshort{it} organisation.
    Dealing with infrastructure provisioning including networking and security can be a tedious task and requires a lot of skills, if done correctly.
    On top of that, it comes not only with \glsdisp{op_ex}{operating}, but also \gls{cap_ex}.
    Moving to the consumer end of the spectrum and leaving the heavy lifting to the big tech giants seems to be effective.
    A \gls{hybrid_cloud_ops} model can help with the move and with becoming efficient.

    \subsubsection{Make use of DevOps}

    \gls{devops} is the culture, the mindset that comes with cloud.
    Hiding complex operations means automation.
    Automation means development, inevitably.
    Google has recognized that long before \gls{devops} and they call it \acrlong{sre}.
    \gls{devops} is also a simple and powerful expression representing communication patterns and best practices.
    Not making use of \gls{devops} basically means ignoring the progress and evolvement of \acrshort{it} over the last decade.
    For this reason, \gls{hybrid_cloud_ops} is built around \gls{devops} pillars and principles.

    \subsubsection{Embrace culture}

    First and foremost, it is important to have a strategy that includes target state and a road-map on how to get there.
    The stages of \gls{devops} may be used as a supporting frame to put that strategy into context.
    Without a clear target state, there is nothing to measure against and it cannot be ensured that the steps taken are putting things on the right track.
    It is important for people to have a plan and know what the expectations are.
    This allows a collaborative culture to form.

    On top of that, a learning and improvement culture should be promoted.
    Change has to come with innovation.
    If innovation is not encouraged, the process of change will be costly and the results might not be satisfying.
    There are many technical and non-technical challenges in a complex enterprise environment.
    This complexity may be unloaded onto the people, but they require the tools and resources to deal with it.
    I believe that complex problems require simple solutions.
    Using a \gls{hybrid_cloud_ops} approach helps to deal with those problems, giving the people the time to do more meaningful work.

    \subsubsection{Harness technology}

    It is strongly recommended to use industry standard, open-source technology and move away from homegrown solutions.
    The technology community has become very open and supportive and many problems have been solved already.
    \gls{kubernetes} is a perfect example, providing an excellent solution to common problems and promoting standardization.
    Using an established framework allows to slowly and steadily move away from legacy components.
    A clear target state helps to remove any ambiguity.
    \gls{hybrid_cloud_ops} leverages open-source technology to deliver the best of \gls{hybrid_cloud} and at the same time supports unifying and moving away from legacy software.
    Because the legacy integration is reusing the \gls{kubernetes} \acrshort{api}, adopting it means working towards the target state and therefore does not extend the existing legacy.
    By making use of \gls{kubernetes} in combination with \gls{hybrid_cloud_ops}, offering hybrid continuous deployments with legacy integration, one can fully profit from industry best practices, with minimal cost and maintenance overhead.

\end{document}

